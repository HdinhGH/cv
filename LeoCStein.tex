\documentclass[margin,line]{res}
\usepackage{hyperref}


\oddsidemargin -.5in
\evensidemargin -.5in
\voffset -25pt
%\topmargin -.2in
\headsep 25pt
\textwidth=6.0in
\textheight=9.3in
\itemsep=0in
\parsep=0in

% Headings
\pagestyle{myheadings}
\markright{Leo C. Stein --- Cirriculum Vitae}

\begin{document}

\newcommand{\myname}{Leo C. Stein}
\newlength{\mynamewidth}
\settowidth{\mynamewidth}{\namefont\myname}

\name{\hspace*{0.5\textwidth}\hspace{-0.5\mynamewidth} \myname \vspace*{.1in}}
% On the first page, have no header.
\thispagestyle{empty}

\begin{resume}

\section{\sc Contact Information}
%\vspace{.05in}
Huy Dinh      \hfill \href{mailto:hdinh@math.utah.edu}{hdinh@math.utah.edu}\\
Department of Mathematics   \hfill \href{http://www.math.utah.edu/~hdinh/}{math.utah.edu/{\raise.17ex\hbox{$\scriptstyle\sim$}}hdinh}\\
155 S 1400 E RM 233  \hfill \href{tel:281-248-3492}{281-248-3492}\\
Salt Lake City, UT 84112-0090 

%%%%%%%%%%%%%%%%%%%%%%%%%%%%%%%%%%%%%%%%%%%%

% Local Variables:
% mode: latex
% TeX-master: "LeoCStein.tex"
% End:


\section{\sc Education}
{\bf Ph.D., Physics,} Massachusetts Institute of Technology, Cambridge, MA, USA \hfill {\bf June 2012}\\
\vspace*{-.1in}
\begin{itemize}
\item[ ] Dissertation Advisor: Prof. Scott Hughes
\item[ ] Dissertation Title: {\it Probes of strong-field gravity}
%\item[ ] G.P.A.: 4.6/5.0
\end{itemize}

{\bf B.S., Physics,} California Institute of
Technology, Pasadena, CA, USA \hfill {\bf June 2006}\\
\vspace*{-.1in}
\begin{itemize}
\item[ ] Degree conferred with honor.
\item[ ] Senior Thesis Advisors: Dr. Patrick Sutton and Prof. Alan Weinstein
%\item[ ] G.P.A.: ?.??/4.00
\end{itemize}

\section{\sc Research Interests}
General relativity (GR), gravitation, and astrophysical phenomena which can
elucidate gravity. Recent work is focused on gravitational-wave predictions
in almost-GR effective theories of gravity. Work in progress and
future work includes numerically solving for gravitational waves in
extreme mass-ratio inspirals and investigating cosmological
signals of gravity theories from the early universe.

\section{\sc Honors and Awards}
{\bf Henry Kendall Teaching Award,} Massachusetts Insitute of Technology \hfill {\bf 2011}\\
\\
{\bf Upperclass Merit Scholarship,} California Insitute of Technology \hfill {\bf 2005--2006}\\

\section{\sc Teaching Experience}
{\bf Guest Lecturer}, Massachusetts Institute of Technology
\vspace*{.05in}  
\begin{itemize}
\item[ ] 8.901, Graduate Astrophysics I \hfill {\bf Spring 2011}
\end{itemize}
{\bf Teaching Assistant}, Massachusetts Institute of Technology
\vspace*{.05in}
\begin{itemize}
\item[ ] 8.942, Cosmology \hfill {\bf Fall 2011}
\item[ ] 8.901, Graduate Astrophysics I \hfill {\bf Spring 2011}
\item[ ] 8.286, The Early Universe \hfill {\bf Fall 2009}
\end{itemize}
{\bf Teaching Assistant}, California Institute of Technology
\vspace*{.05in}
\begin{itemize}
\item[ ] Ph 7, Nuclear and Quantum Physics Lab\hfill {\bf Spring 2005}
\item[ ] Ph 5, Analog Electronics for Physicists \hfill {\bf Fall 2004}
\end{itemize}

% \section{\sc Publications in Preparation}
% \begin{enumerate}
% \item[{1.}] {\bf Stein,~L.~C.}, Yagi,~K. (2013)
%   {\it Constraining and parameterizing scalar corrections to general relativity}.
% \end{enumerate}

% \section{\sc Submitted Publications}
% \begin{enumerate}
% \item[{1.}] {\bf Stein,~L.~C.}, Yagi,~K., Yunes,~N. (2013)
%   {\it Three-Hair Newtonian Relations for Rotating Stars},
%   \href{http://arxiv.org/abs/1312.4532}{arXiv:1312.4532}. ApJ
%   accepted, in press.
% \end{enumerate}

% \section{\sc Publications Accepted for Publication}
% \begin{enumerate}
% \item[{1.}] {\bf Stein,~L.~C.}, Yagi,~K. (2013)
%   {\it Constraining and parameterizing scalar corrections to general relativity},
%   \href{http://arxiv.org/abs/1310.6743}{arXiv:1310.6743}. Submitted to PRD.
% \end{enumerate}

\section{\sc Refereed Publications}
\begin{enumerate}
\item[{10.}] {\bf Stein,~L.~C.}, Yagi,~K., Yunes,~N. (2014)
  {\it Three-Hair Newtonian Relations for Rotating Stars},
  \href{http://dx.doi.org/10.1088/0004-637X/788/1/15}{ApJ~{\bf 788}~15}
  [\href{http://arxiv.org/abs/1312.4532}{arXiv:1312.4532}]
\item[{9.}] {\bf Stein,~L.~C.}, Yagi,~K. (2013)
  {\it Constraining and parameterizing scalar corrections to general relativity},
  \href{http://arxiv.org/abs/1310.6743}{Phys.~Rev.~D {\bf 89} 044026}
  [\href{http://arxiv.org/abs/1310.6743}{arXiv:1310.6743}]
\item[{8.}] Yagi,~K., {\bf Stein,~L.~C.}, Yunes,~N., Tanaka,~T. (2013)
  {\it Isolated and Binary Neutron Stars in Dynamical Chern-Simons
    Gravity},
  \href{http://arxiv.org/abs/1302.1918}{Phys.~Rev.~D {\bf 87} 084058}
  [\href{http://arxiv.org/abs/1302.1918}{arXiv:1302.1918}]
\item[{7.}] Yagi,~K., {\bf Stein,~L.~C.}, Yunes,~N.,
  Tanaka,~T. (2012), {\it Post-Newtonian, Quasi-Circular Binary
    Inspirals in Quadratic Modified Gravity},
  \href{http://arxiv.org/abs/1110.5950}{Phys.~Rev.~D {\bf 85} 064022}
  [\href{http://arxiv.org/abs/1110.5950}{arXiv:1110.5950}]
\item[{6.}] Vigeland,~S., Yunes,~N., {\bf Stein,~L.~C.} (2011), {\it
    Bumpy black holes in alternative theories of gravity},
  \href{http://arxiv.org/abs/1102.3706}{Phys.~Rev.~D {\bf 83} 104027}
  [\href{http://arxiv.org/abs/1102.3706}{arXiv:1102.3706}]
\item[{5.}] Yunes,~N., {\bf Stein,~L.~C.} (2011), {\it Nonspinning
    black holes in alternative theories of gravity},
  \href{http://arxiv.org/abs/1101.2921}{Phys.~Rev.~D {\bf 83} 104002}
  [\href{http://arxiv.org/abs/1101.2921}{arXiv:1101.2921}]
\item[{4.}] {\bf Stein,~L. C.}, Yunes,~N. (2011), {\it Effective
    gravitational wave stress-energy tensor in alternative theories of
    gravity},
  \href{http://arxiv.org/abs/1012.3144}{Phys.~Rev.~D {\bf 83} 064038}
  [\href{http://arxiv.org/abs/1012.3144}{arXiv:1012.3144}]
\item[{3.}] Lutomirski,~A., Tegmark,~M., Sanchez,~N.~J., {\bf
    Stein,~L.~C.},
  Urry,~W.~L., Zaldarriaga,~M. (2011), {\it Solving the
    corner-turning problem for large interferometers},
  \href{http://arxiv.org/abs/0910.1351}{MNRAS {\bf 410} 2075}
  [\href{http://arxiv.org/abs/0910.1351}{arXiv:0910.1351}]
\item[{2.}] Sutton,~P., Jones,~G., Chatterji,~S., Kalmus,~P., Leonor,~I.,
  Poprocki,~S., Rollins,~J., Searle,~A., {\bf Stein,~L.}, Tinto,~M.,
  Was,~M. (2010), {\it X-Pipeline: an analysis package for autonomous
    gravitational-wave burst searches},
  \href{http://arxiv.org/abs/0908.3665}{New J. Phys. {\bf 12} 053034}
  [\href{http://arxiv.org/abs/0908.3665}{arXiv:0908.3665}]
\item[{1.}] Chatterji,~S., Lazzarini,~A., {\bf Stein,~L.}, Sutton,~P.,
  Searle,~A. (2006), {\it Coherent network analysis technique for
    discriminating gravitational-wave bursts from instrumental noise},
  \href{http://arxiv.org/abs/gr-qc/0605002}{Phys.~Rev.~D {\bf 74}, 082005}
  [\href{http://arxiv.org/abs/gr-qc/0605002}{arXiv:gr-qc/0605002}]
\end{enumerate}

\section{\sc Unrefereed Publications}
\begin{enumerate}
\item[{4.}] {\bf Stein,~L.~C.} (2012), {\it Probes of Strong-field Gravity}, Ph.D. thesis at Massachusetts Institute of Technology
  [\href{http://hdl.handle.net/1721.1/77256}{hdl:1721.1/77256}]
\item[{3.}] Betancourt,~M., {\bf Stein,~L.~C.} (2011) {\it The
    Geometry of Hamiltonian Monte Carlo},
  [\href{http://arxiv.org/abs/1112.4118}{arXiv:1112.4118}]
\item[{2.}] {\bf Stein,~L.~C.} (2009), {\it Binary Inspiral
    Gravitational Waves from a Post-Newtonian Expansion}, Contribution
  to the Wolfram Demonstrations Project, \url{http://demonstrations.wolfram.com/BinaryInspiralGravitationalWavesFromAPostNewtonianExpansion/}
\item[{1.}] {\bf Stein,~L.~C.} (2006), {\it Gravitational Wave Burst Source Localization in a Coherent Network Analysis}, Senior thesis at California Institute of Technology
\end{enumerate}

%%%%%%%%%%%%%%%%%%%%%%%%%%%%%%%%%%%%%%%%%%%%

% Local Variables:
% mode: latex
% TeX-master: "LeoCStein.tex"
% End:


\section{\sc Invited Talks}
\begin{enumerate}
\item[{2.}] {\it Signatures of strong gravity corrections to GR},
  Cornell Relativity Lunch, November 2011
\item[{1.}] {\it Signatures of strong gravity corrections to GR},
  MSU RelAstro Seminar, October 2011
\end{enumerate}

\section{\sc Contributed Talks}
\begin{enumerate}
%\item[{4.}] {\it Binary inspirals in quadratic gravity theories}, Cornell relativity lunch, November 2011
\item[{3.}] {\it Effective gravitational wave stress-energy tensor in
    alternative theories of gravity}, Eastern Gravity Meeting, June 2011
\item[{2.}] {\it Effective gravitational wave stress-energy tensor in alternative theories of gravity}, April APS Meeting 2011
\item[{1.}] {\it Tuning advanced gravitational-wave detectors to
    optimally measure neutron-star merger waves}, April APS Meeting 2010
\end{enumerate}

\section{\sc Professional Activities}
{\bf Member, American Physical Society}\hfill{\bf 2010--Present}
\vspace*{.05in}  
\begin{itemize}
\item[] Division of Astrophysics
\item[] Topical Group on Gravitation
\end{itemize}

\section{\sc Service}
{\bf Co-organizer}
\vspace*{.05in}
\begin{itemize}
\item[] General Relativity Informal Tea-Time Series (GRITTS)\hfill {\bf Fall 2011--Present}
\item[] MKI Journal Club\hfill {\bf Fall 2007--Spring 2010}
\end{itemize}

\end{resume}
\end{document}




