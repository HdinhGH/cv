\documentclass[margin,line]{res}
\usepackage{hyperref}


\oddsidemargin -.5in
\evensidemargin -.5in
\topmargin -.2in
\textwidth=6.0in
\textheight=9.5in
\itemsep=0in
\parsep=0in

\newenvironment{list1}{
  \begin{list}{\ding{113}}{%
      \setlength{\itemsep}{0in}
      \setlength{\parsep}{0in} \setlength{\parskip}{0in}
      \setlength{\topsep}{0in} \setlength{\partopsep}{0in} 
      \setlength{\leftmargin}{0.17in}}}{\end{list}}
\newenvironment{list2}{
  \begin{list}{$\bullet$}{%
      \setlength{\itemsep}{0in}
      \setlength{\parsep}{0in} \setlength{\parskip}{0in}
      \setlength{\topsep}{0in} \setlength{\partopsep}{0in} 
      \setlength{\leftmargin}{0.2in}}}{\end{list}}


\begin{document}

\newcommand{\myname}{Leo C. Stein}
\newlength{\mynamewidth}
\settowidth{\mynamewidth}{\namefont\myname}

\name{\hspace*{0.5\textwidth}\hspace{-0.5\mynamewidth} \myname \vspace*{.1in}}

\begin{resume}
\section{\sc Contact Information}
\vspace{.05in}
%\begin{tabular}{@{}p{4in}p{4in}}
Dept. of Physics and MIT Kavli Institute            \hfill 585-729-5898 \\
Bldg. 37, Rm. 602, 70 Vassar Street    \hfill \href{mailto:leostein@mit.edu}{leostein@mit.edu}\\
Cambridge, MA 02139 USA \hfill \href{http://web.mit.edu/leostein/www/}{web.mit.edu/leostein/www/}\\
%\end{tabular}




\section{\sc Education}
{\bf Ph.D., Physics,} Massachusetts Institute of Technology, Cambridge, MA, USA \hfill {\bf June 2012}\\
\vspace*{-.1in}
\begin{itemize}
\item[ ] Dissertation Advisor: Prof. Scott Hughes
\item[ ] Dissertation Title: {\it Probes of strong-field gravity}
%\item[ ] G.P.A.: 4.6/5.0
\end{itemize}

{\bf B.S., Physics,} California Institute of
Technology, Pasadena, CA, USA \hfill {\bf June 2006}\\
\vspace*{-.1in}
\begin{itemize}
\item[ ] Degree conferred with honor.
\item[ ] Senior Thesis Advisors: Dr. Patrick Sutton and Prof. Alan Weinstein
%\item[ ] G.P.A.: ?.??/4.00
\end{itemize}

\section{\sc Research Interests}
General relativity (GR), gravitation, and astrophysical phenomena which can
elucidate gravity. Recent work is focused on gravitational-wave predictions
in almost-GR effective theories of gravity. Work in progress and
future work includes numerically solving for gravitational waves in
extreme mass-ratio inspirals and investigating cosmological
signals of gravity theories from the early universe.

\section{\sc Honors and Awards}
{\bf Henry Kendall Teaching Award,} Massachusetts Insitute of Technology \hfill {\bf 2011}\\
\\
{\bf Upperclass Merit Scholarship,} California Insitute of Technology \hfill {\bf 2005--2006}\\

\section{\sc Teaching Experience}
{\bf Guest Lecturer}, Massachusetts Institute of Technology
\vspace*{.05in}  
\begin{itemize}
\item[ ] 8.901, Graduate Astrophysics I \hfill {\bf Spring 2011}
\end{itemize}
{\bf Teaching Assistant}, Massachusetts Institute of Technology
\vspace*{.05in}
\begin{itemize}
\item[ ] 8.901, Graduate Astrophysics I \hfill {\bf Spring 2011}
\item[ ] 8.286, The Early Universe \hfill {\bf Fall 2009}
\end{itemize}
{\bf Teaching Assistant}, California Institute of Technology
\vspace*{.05in}
\begin{itemize}
\item[ ] Ph 7, Nuclear and Quantum Physics Lab\hfill {\bf Spring 2005}
\item[ ] Ph 5, Analog Electronics for Physicists \hfill {\bf Fall 2004}
\end{itemize}

\section{\sc Publications in Preparation}
\begin{enumerate}
\item[{1.}] Yagi,~K., {\bf Stein,~L.~C.}, Yunes,~N.,
  Tanaka,~T. (2011), {\it Binary inspirals in quadratic gravity theories}
\end{enumerate}

\section{\sc Refereed Publications}
\begin{enumerate}
\item[{6.}] Vigeland,~S., Yunes,~N., {\bf Stein,~L.~C.} (2011), {\it
    Bumpy black holes in alternative theories of gravity},
  Phys. Rev. D {\bf 83} 104027
\item[{5.}] Yunes,~N., {\bf Stein,~L.~C.} (2011), {\it Nonspinning
    black holes in alternative theories of gravity},
  Phys. Rev. D {\bf 83} 104002
\item[{4.}] {\bf Stein,~L. C.}, Yunes,~N. (2011), {\it Effective
    gravitational wave stress-energy tensor in alternative theories of
    gravity}, Phys. Rev. D {\bf 83} 064038
\item[{3.}] Lutomirski,~A., Tegmark,~M., Sanchez,~N.~J., {\bf
    Stein,~L.~C.},
  Urry,~W.~L., Zaldarriaga,~M. (2011), {\it Solving the
    corner-turning problem for large interferometers}, MNRAS {\bf 410} 2075
\item[{2.}] Sutton,~P., Jones,~G., Chatterji,~S., Kalmus,~P., Leonor,~I.,
  Poprocki,~S., Rollins,~J., Searle,~A., {\bf Stein,~L.}, Tinto,~M.,
  Was,~M. (2010), {\it X-Pipeline: an analysis package for autonomous
    gravitational-wave burst searches}, New J. Phys. {\bf 12} 053034
\item[{1.}] Chatterji,~S., Lazzarini,~A., {\bf Stein,~L.}, Sutton,~P.,
  Searle,~A. (2006), {\it Coherent network analysis technique for
    discriminating gravitational-wave bursts from instrumental noise},
  Phys. Rev. D {\bf 74}, 082005
\end{enumerate}

\section{\sc Unrefereed Publications}
\begin{enumerate}
\item[{2.}] {\bf Stein,~L.~C.} (2009), {\it Binary Inspiral
    Gravitational Waves from a Post-Newtonian Expansion}, Contribution
  to the Wolfram Demonstrations Project, \url{http://demonstrations.wolfram.com/BinaryInspiralGravitationalWavesFromAPostNewtonianExpansion/}
\item[{1.}] {\bf Stein,~L.~C.} (2006), {\it Gravitational Wave Burst Source Localization in a Coherent Network Analysis}, Senior thesis at California Institute of Technology
\end{enumerate}

\section{\sc Contributed Talks}
\begin{enumerate}
%\item[{4.}] {\it Binary inspirals in quadratic gravity theories}, Cornell relativity lunch, November 2011
\item[{3.}] {\it Effective gravitational wave stress-energy tensor in
    alternative theories of gravity}, Eastern Gravity Meeting, June 2011
\item[{2.}] {\it Effective gravitational wave stress-energy tensor in alternative theories of gravity}, April APS Meeting 2011
\item[{1.}] {\it Tuning advanced gravitational-wave detectors to
    optimally measure neutron-star merger waves}, April APS Meeting 2010
\end{enumerate}

\section{\sc Professional Activities}
{\bf Member, American Physical Society}\hfill{\bf 2010--Present}
\vspace*{.05in}  
\begin{itemize}
\item[] Division of Astrophysics
\item[] Topical Group on Gravitation
\end{itemize}

\section{\sc Service}
{\bf Co-organizer}
\vspace*{.05in}
\begin{itemize}
\item[] General Relativity Informal Tea-Time Series (GRITTS)\hfill {\bf Fall 2011--Present}
\item[] MKI Journal Club\hfill {\bf Fall 2007--Spring 2010}
\end{itemize}

\end{resume}
\end{document}




